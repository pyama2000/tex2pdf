\documentclass[twocolumn,9]{ltjsarticle}

\usepackage[top=15mm,bottom=15mm,left=20mm,right=20mm]{geometry}
\usepackage[haranoaji,nfssonly]{luatexja-preset}
\usepackage{graphicx}

\title{IRC, P2PおよびHTTPを用いたボットネットの検出}
\title{周波数分析によるワームの周期性特性について}
\author{19FMI29 山下 尚彦 \\ 指導教員: 寺田 真敏}
\date{}

\begin{document}
\maketitle

\section{はじめに}
本調査レポートは4章で構成されており, 2章で調査した論文名とその概要について述べる. 
3章では論文の新規性や有効性を評価し, 4章でそれに対する考察を行う. 

\section{調査内容}
本章では調査した論文について述べる. 

\subsection{調査論文名}
今回調査した論文は仲小路らの論文「周波数分析に基づくインシデント傾向検知手法に関する検討」
\cite{仲小路博史2005周波数分析に基づくインシデント傾向検知手法に関する検討}である. 

\subsection{論文概要}
不正アクセスツールの高機能化や利用の拡大によって企業や組織は, これらの脅威からネットワークや情報資源を守るために, 
侵入検知システム(IDS)やファイアウォールなどの脅威を捉えるシステムを導入している. 
しかし, これらのシステムが出力するイベントログが膨大なことやそれらを分析するためには専門的な知識が必要なことなどから
イベントログから脅威を的確に発見することは難しい. 

そこで仲小路らはイベントログから, 不正アクセスツールやマルウェアのルーチンワークや生活リズムなどの要因による
周期性に基づいた特徴づけを行い, 代表的なワーム(マルウェアの一種)の周期的な特性の数値化を試みた. 

\section{新規性(オリジナリティ), 有効性の評価方法}
仲小路らの実機検証で使われたワームは, Nimda.E, SQLSlammer, MSBlaster, Sasser.Aの4つである. 
Nimda.Eは80番と445番ポートの脆弱性を利用して感染し, Sasser.Cも同様に445番ポートの脆弱性を利用するために
ネットワークスキャンを行う. MSBlasterとSQLSlammerはそれぞれ135番と1434番ポートを利用して感染拡大を図ろうとする. 
それぞれのワームが感染したネットワークのイベントログに対して高速フーリエ変換(FFT)による周期的な特徴の抽出を行った結果, 
SQLSlammer以外のワームで周期性を確認できた. 

仲小路らの考察では, SQLSlammerで周期性を確認できなかったのは, このワームが短時間で大量にパケットを送信するため
FFTで利用するサンプリングレートが不足していたことと, UDPパケットを用いていたことによって再送処理が
行われなかったことが原因であるとしている. 

\section{考察}
SQLSlammerで周期性を確認できなかったのはUDPパケットを用いていることが原因の一つだと仲小路らは考察したが, 
ワームの通信をイベントログでは観測できているので, FFTによる分析で周期的な特徴を表してもおかしくないのではないかと
疑問に思った.

一方で, SQLSlammerはシステムのリソースが許す限り短時間で大量のパケットを送信するため, FFTでは
周期性が観測できなかったとしている. そのような通信特性を持つ他のマルウェアに対しても同様に高速フーリエ変換では
周期性を観測することはできないのではないかと考える. 
私の研究ではLomb-Scargleピリオドグラムによる周波数解析を行ってボットネットの周期的な特徴を検出する手法を提案したが, 
SQLSlammerのような通信特性を持ったボットネットに対してどのような結果を示すか疑問に思った. 

\bibliographystyle{junsrt}
\bibliography{DB}
\end{document}