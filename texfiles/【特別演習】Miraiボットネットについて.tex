\documentclass[twocolumn,9]{ltjsarticle}

\usepackage[top=15mm,bottom=15mm,left=20mm,right=20mm]{geometry}
\usepackage[haranoaji,nfssonly]{luatexja-preset}
\usepackage{graphicx}

\title{Miraiボットネットについて}
\author{19FMI29 山下 尚彦 \\ 指導教員: 寺田 真敏}
\date{}

\begin{document}
\maketitle

\section{はじめに}
本調査レポートは4章で構成されており, 2章で調査した論文名とその概要について述べる. 
3章では論文の新規性や有効性を評価し, 4章でそれに対する考察を行う. 

\section{調査内容}
本章では調査した論文について述べる. 

\subsection{調査論文名}
今回調査した論文はAntonakakisらの論文「Understanding the Mirai Botnet」
\cite{antonakakis2017understanding}である. 

\subsection{論文概要}
組み込み機器やIoT機器の普及率は急速に伸びているのに対して、そのセキュリティやプライバシー、安全性に関しては
疎かになっている傾向がある. 
Miraiボットネットは、そのようなセキュリティに問題のある組み込み機器やIoT機器で構成されたボットネットで、
2016年後半にMiraiボットネットによって発生した大規模な分散型サービス拒否(DDoS)攻撃で注目されるようになった.  

そこで, AntonakakisらはMiraiのの出現と亜種の進化, 標的にされたデバイス, そして実行された攻撃について分析を行った. 
調査の結果, 感染時のC2サーバに接続する手順やネットワーク内で感染を広げるプロセス, 高度な攻撃手法などが明らかになった. 

\section{Miraiボットネット感染初期について}
Antonakakisらはインターネットに公開されたMiraiボットネットのソースコードからその構造と感染, 攻撃にいたる手順を
分析した. その分析結果の中から特に感染から感染初期の挙動について本調査レポートで紹介する. 

前提として, Miraiに感染したルータやIoTデバイスがネットワーク内に存在するとする. すでに感染したデバイスは
さらにネットワーク内の他の機器に感染を広げようと, ランダムに生成したIPv4アドレスのTelnetのTCPポート23と2323に対して
接続を試みる. その際にMiraiが保持している認証情報を組み合わせてTelnetの接続を確立しようとする. 
1回目のログインで成功すると, 被害者のIPアドレスと認証情報を攻撃者が用意したサーバに送信し, 今度はそのサーバから
被害者の機器にログインして, 機器のシステム環境などに応じたマルウェアをダウンロードする. そして、ダウンロードした
マルウェアを実行しMiraiに感染させる. 
感染後は, Mirai以外のマルウェアの活動によって自身が発見されないように競合するマルウェアのプロセスを停止し, 
C2サーバから攻撃命令を待ち受けたり, 他の機器に感染を広げるために再度ネットワークをスキャンしたりする. 

Antonakakisらは他にもMiraiの攻撃プロセスやどのようなIoTデバイスが標的にされやすいかや感染が拡大している地域
などについても調査を行っていて, それらの調査結果を基に対策例などを提示した. 

\section{考察}
AntonakakisらのMiraiボットネットに関する調査は, 感染方法や感染拡大のプロセス, 攻撃手法やその亜種の進化について
詳細に調査されており, 私の研究においてボットネットを特定するのに有効な情報が多かった. 
特に, 感染初期のC2サーバとの通信や感染拡大のプロセスについては私の研究で使用する部分なので, 
実際に世界的に注目されたボットネットの挙動を確認できたのは良かったと思う. 

\bibliographystyle{junsrt}
\bibliography{DB}
\end{document}