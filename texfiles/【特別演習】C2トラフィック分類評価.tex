\documentclass[twocolumn,9]{ltjsarticle}

\usepackage[top=15mm,bottom=15mm,left=20mm,right=20mm]{geometry}
\usepackage[haranoaji,nfssonly]{luatexja-preset}
\usepackage{graphicx}

\title{C2トラフィック分類評価}
\author{19FMI29 山下 尚彦 \\ 指導教員: 寺田 真敏}
\date{}

\begin{document}
\maketitle

\section{はじめに}
本調査レポートは4章で構成されており, 2章で調査した論文名とその概要について述べる. 
3章では論文の新規性や有効性を評価し, 4章でそれに対する考察を行う. 

\section{調査内容}
本章では調査した論文について述べる. 

\subsection{調査論文名}
今回調査した論文は山内らの論文「C\&Cトラフィック分類のための機械学習手法の評価」
\cite{山内一将2015c}である. 

\subsection{論文概要}
インターネットの普及にともない, ボットネットによる脅威が増している. 
特に攻撃者がボットネットを運用するために利用されるCommand\&Controlサーバ(C2サーバ)を特定する手法が必要とされる. 
C2サーバは攻撃者の命令をボットに感染した端末群に一斉に転送し, 命令を受けた端末群は命令の内容に従ってDDoS攻撃や
スパムメールの送信, 脆弱性スキャン攻撃などを行う. この攻撃者の命令を転送するC2サーバの通信を検出することで, 
大規模な攻撃を未然に防ぐ事ができると山内らは考えた. 

そこで, 山内らはクライアントとサーバの通信の区別とクライアントがサーバにアクセスする挙動を表現したものとして定義した
新たな特徴ベクトルを提案し, C2サーバの特定する手法を提案した. 

\section{有効性の評価方法}
ボットネットを制御するC2サーバとボット間の通信手段は大きく分けて2つの形態を取り, 1つ目はIRCを用いたC2サーバから
ボットに対する一方的な通信である. 2つめはHTTPやP2Pを用いてC2サーバとボットが双方向に通信を行うもので, 
特にHTTPの通信量はIRCに比べて大きく, 異常な通信のみを正確に取り出すことが困難であるためHTTP型ボットネットは
増加の傾向がある. 

そこで, 山内らは多様なプロトコルを利用するC2サーバを特定するための新たな特徴ベクトルを提案し, SVM, 
ロジスティック回帰(LR), ナイーブベイズ(NB)の機械学習を用いて, それぞれの分類精度と実行時間で評価を行った. 
SVMを用いたC2トラフィックの検出ではHTTP, IRCともに98\%以上の検出できた. しかし, LRとNBではデータセットによっては
HTTPを用いた通信の検出率が85\%程度の精度となった. また, 誤検知率はそれぞれIRCで17.0\%, 25.8\%, 25.4\%と
全体的に高く, HTTPでは2.8\%, 27.0\%, 10.3\%となり, LRはIRC, HTTPともに誤検知率が高かった. 
実行速度に関してはNBが一番早く, SVMとLRに比べて約20\%の速度で実行することができた. 

\section{考察}
山内らの提案した手法は, ネットワーク全体のトラフィックからC2サーバのみのトラフィックを検出でき, かつ, 
ボットネットのライフサイクルがどの段階であっても特定できる手法となっている. 一方私の研究では, ボットネットが
感染後, 攻撃命令を待っている間のC2サーバからの生存確認やアップデートのための周期的な通信に限定して特定を行っている. 
さらに, C2トラフィック検出の精度もかなり高く, 機械学習による分類は興味深いものがあった. 

私が提案する手法でもC2サーバの通信を特定することができたが, その手法はLomb-Scargleピリオドグラムを用いた
周波数解析による検出なので, 山内らの提案手法とは全く違う. そこで, 山内らの実験結果を私の提案する手法と比較することで, 
有効性や優位性, 課題などを評価する指標としたい. 

\bibliographystyle{junsrt}
\bibliography{DB}
\end{document}