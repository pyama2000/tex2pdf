\documentclass[twocolumn,9]{ltjsarticle}

\usepackage[top=15mm,bottom=15mm,left=20mm,right=20mm]{geometry}
\usepackage[haranoaji,nfssonly]{luatexja-preset}
\usepackage{graphicx}

\title{IRC, P2PおよびHTTPを用いたボットネットの検出}
\author{19FMI29 山下 尚彦 \\ 指導教員: 寺田 真敏}
\date{}

\begin{document}
\maketitle

\section{はじめに}
本調査レポートは4章で構成されており, 2章で調査した論文名とその概要について述べる. 
3章では論文の新規性や有効性を評価し, 4章でそれに対する考察を行う. 

\section{調査内容}
本章では調査した論文について述べる. 

\subsection{調査論文名}
今回調査した論文はAsSadhanらの論文「Analysis of P2P, IRC and HTTP traffic for botnets detection」
\cite{assadhan2018analysis}である. 

\subsection{論文概要}
ボットネットを管理するC2サーバは, ボットネットに対して攻撃命令やボットネット自体の更新, 可用性の確認のために
周期的に通信を行う. そこでAsSadhanらはネットワーク上の通信を監視および分析を行うことでボットネットとC2サーバ間の
通信の検出する手法を提案した. 通信の分析を行う上でASSadhanらは複数のピリオドグラムから有効性を評価し, 
ボットネットを検出するアプローチを確立した. 

実験では実際に大学内のネットワークを50日間キャプチャし, 通信にIRC, P2PおよびHTTPを用いたボットネットの
検出を試みた. 実験の結果, ボットネットを検出することができ, その通信周期は31分から49分の間であることがわかった. 

\section{新規性(オリジナリティ), 有効性の評価方法}
AsSadhanらはボットネットとC2サーバが通信を行う際に使用するプロトコルに注目してボットネットの検出を行った. 
ボットネットとC2サーバ間の通信はIRC, P2PおよびHTTPプロトコルが用いられており, それぞれの規定のポート番号である
6667, 11375, 80番ポートを使用する通信に限定して分析を行った. その結果, IRCとP2Pのポート上の通信で
それぞれ33分, 49分周期で通信を行うボットネットを検出することができた. しかし, HTTPを使用したボットネットに関しては
80番ポートを使用する他のアプリケーションの通信が膨大な量であったため, 検出することができなかった. 
そこで, 80番ポート上の通信で疑わしいホストと通信を行っているものと内部ホスト間の通信でフィルタリングして再度分析を行った. 
その結果, 31分周期でC2サーバと通信を行っているボットネットを検出することができた. 

\section{考察}
AsSadhanらの研究ではポート番号を限定してボットネットを検出したが, 規定のポート番号を利用しないボットネットによる通信
の場合全く検出することができない問題点がある. 
また, 80番ポートを分析する際に疑わしいホストとの通信でフィルタリングを行っていたが, どのようにフィルタリングを行ったかが
言及されていなかったため, どのような手法によって絞り込むことができたのかが疑問である. 

私の研究では, AsSadhanらが行った実験ほど通信量がないため送信元IPアドレスと受信先IPアドレスのペアをピリオドグラムを
用いて分析することができたが, 膨大な通信量が記録されたキャプチャファイルに対して分析を行うと現実的な時間で分析することが
できない可能性がある. 
そこで, AsSadhanらのように分析するポート番号の範囲を決めることでその問題点を解決することができると考えられる. 


\bibliographystyle{junsrt}
\bibliography{DB}
\end{document}