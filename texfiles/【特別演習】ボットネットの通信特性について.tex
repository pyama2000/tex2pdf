\documentclass[twocolumn,9]{ltjsarticle}

\usepackage[top=15mm,bottom=15mm,left=20mm,right=20mm]{geometry}
\usepackage[haranoaji,nfssonly]{luatexja-preset}
\usepackage{graphicx}

\title{ボットネットの通信特性について}
\author{19FMI29 山下 尚彦 \\ 指導教員: 寺田 真敏}
\date{}

\begin{document}
\maketitle

\section{はじめに}
本調査レポートは4章で構成されており, 2章で調査した論文名とその概要について述べる. 
3章では論文の新規性や有効性を評価し, 4章でそれに対する考察を行う. 

\section{調査内容}
本章では調査した論文について述べる. 

\subsection{調査論文名}
今回調査した論文はVormayrらの論文「Botnet Communication Patterns」
\cite{vormayr2017botnet}である. 

\subsection{論文概要}
ボットネットの脅威に対して, 今日までにその通信構造や制御プロトコル, ボットネットの目的などの調査・分類が行われ, 
ボットネットの検出技術は様々に提案されている. 一方で, ボットネットの構築や運用も同様に進化しており, その結果
これまで知られていなかったタイプのボットネットの検出方法が必要とされている. 

そこでVormayrらはボットネットの構築や運用などのネットワーク周辺について分析し, 
今後の研究のために調査結果を論文にまとめた. 

\section{ボットネットの分析結果について}
ボットネットは複数のボットからなる端末群で, 攻撃者の命令を実行するためにC2サーバが必要となる. 
このC2サーバとボットを通信の観点からみた関係には, C2サーバが1台の中央集権型, P2Pネットワークを用いた分散型, 
中央集権型と分散型を合わせたハイブリッド型がある. また, その通信に使われるプロトコルにも様々なものがあり, 
従来は実装が容易で遅延が少ないIRCが用いられることが多かった. 
しかし, IRCプロトコルは企業によっては許可されていなかったりファイアウォールでブロックされたりと
一般的な通信プロトコルではないため, 一般的によく使われるHTTPプロトコルを用いることが増えた. 
HTTPの他に, 家庭や企業のネットワークでよく使われるプロトコルであるSMBを利用して, 通信を秘匿するボットネットも登場した. 
分散型のボットネットでは主にP2Pプロトコルを用いてボット同士で通信を行う. 
また, 既存のプロトコルではなく全く新しい通信プロトコルを実装したボットネットも存在しており, 
多くはUDPやTCP, ICMPをベースとしたものである. 

Vormayrは他にもボットネットトラフィックの隠蔽や難読化の技術や通信パターンなどの調査を行っており, 
今後のボットネットのネットワーク関連, 特に通信パターンによる検出に役立つような分析結果をまとめた. 

\section{考察}
Vormayrらの分析により, 最近のボットネットの構造や通信に利用するプロトコルの特徴, さらにC2サーバとボット間の
通信パターン・通信メッセージの詳細な情報がわかった. これにより私の研究において, 調査によって明らかになったボットネットの
通信特性を考慮して, より精度の高いボットネット検出手法を提案することができると考えられる. 

\bibliographystyle{junsrt}
\bibliography{DB}
\end{document}