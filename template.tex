\documentclass[twocolumn,10pt]{ltjsarticle}

\usepackage[top=20mm,bottom=20mm,left=25mm,right=25mm,columnsep=10mm]{geometry}
\usepackage[haranoaji,nfssonly]{luatexja-preset}
%% 画像を使う場合は以下のコメントを外す
% \usepackage{graphicx}
%% URLを使う場合は以下のコメントを外す
% \usepackage{url}

\title{\TeX とは}
\author{筱 更治}
\date{\today}

\begin{document}
\maketitle

\section{\TeX について}

TeXはスタンフォード大学教授(数学)D.E.Knuth(19388~)による文書整形システムです。TeXは大抵「テフ」と読まれいます。TeXはワープロのたぐいと言えますが、より正しくは、1つのプログラム言語に近いものです。利用者によるマクロ命令によって機能を拡張することができます。今までは研究者の間でUNIX環境での稼働が一般的でしたが、今日では、個人がMacintoshOSやWindows9Xをインストールしたパーソナルコンピュータ上でTeXを動かすことが可能です。ネットワークで配布されているパッケージもありますが、最近では、安価にCD-ROMの形態で書籍に付録されているものもあり、ある程度の文法の理解は必要ですが、文書作成の種類や目的によっては、とても重宝なツールと言えます。……以下続く……

\bibliographystyle{junsrt}
\bibliography{DB}
\end{document}
